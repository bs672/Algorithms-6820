\documentclass{hmcpset}
\usepackage{amsmath}
\usepackage{amsthm}
\usepackage{algorithm, algpseudocode}
\usepackage{enumitem}
\newtheorem{lemma}{Lemma}
\name{Andrew Bennett, Richard Strong Bowen}
\class{CS6820}
\assignment{Problem Set 1}
\duedate{15 September 2017}
\begin{document}
\begin{problem}[1]
The half-graph is a bipartite graph with vertex sets $L_0 = \{u_1,\ldots,u_n\}$ and $R_0 = \{v_1,\ldots,v_n\}$ and edge set $E = \{(u_i, v_j) | i \leq j\}$.
\begin{enumerate}
\item Prove that the half-graph has a unique perfect matching.
\item Let $G = (L,R,E)$ be a bipartite graph with $|L| = |R| = n$. Suppose $G$ has a unique perfect matching. Prove that $G$ is isomorphic to a subgraph of the half-graph. Your proof should be polynomial-time constructive.
\item If $\pi$ is a permutation of ${1,\ldots,n},$, let $H_\pi$ denote the "permuted half graph" with vertex sets $L_0$ and $R_0$ as in part (a) and with edge set $E_\pi = \{(u_i, v_{\pi(j)} | i \leq j\}$. If ALG is any algorithm for the online matching problem -- potentially randomized -- let $M_\pi$ denote the (potentially random) matching that results from running ALG on the input instance $H_\pi$ with the vertices of $L_0$ arriving in the order $u_1, u_2, \ldots, u_n$. Prove that the expected cardinality of $M_\pi$ is at most $(1-\frac{1}{e} + o(1))n$, where the $o(1)$ term tends to zero as $n\rightarrow \infty$.
\end{enumerate}
\end{problem}
\begin{solution}
\begin{enumerate}[label=(\Alph*)]
  \item 

First it is obvious that there exists a perfect matching for the half graph, since
$M = \{(u_1,v_1),\ldots,(u_n,v_n)\}$ is a matching containing every vertex in $G$.
It only remains to be shown that this matching is unique.

The uniqueness of the matching can be shown via induction. Let $P(n)$ be the
proposition that the half-graph of size $n$ has a unique perfect matching,
as described above. $P(1)$ is clearly true, because with $n = 1$ there are only
two vertices with a single edge, so uniqueness of the perfect matching here
is trivial.

Now suppose that proposition $P(k - 1)$ is true. Consider the half graph $G_k$
containing $2k$ vertices. By construction $u_k$ has a single neighbor in $G_k$,
which is $v_k$. This means any perfect matching must involve $u_k$ and $v_k$
being matched together. Therefore $G_k$ has a unique perfect matching if and
only if the remaining graph after removing $u_k$, $v_k$, and any adjacent edges
has a unique perfect matching. Let us call this smaller graph containing
$2k -2$ vertices $H_k$.

We can observe that $H_k$ is identical to the half graph of size $k - 1$. This
is because it contains vertices $\{u_1,\ldots,u_{k-1},v_1,\ldots,v_{k-1}\}$ with
edges satisfying the same equations as for the half graph (the edge set is
$E = \{(u_i, v_j) | i \leq j\}$). Therefore we know $H_k$ has a unique perfect
matching given proposition $P(k - 1)$ which we have assumed to be true.
So $P(k - 1) \implies P(k)$.

We have proven the base case and the inductive case, so we have our desired
result that the half graph has a unique perfect matching for every $n \in \mathbb{N}$.

\item

Before we present our algorithm, we will prove two preliminary lemmas, that motivate
its construction. Given a graph $G = (V, E)$ and a set of vertices $S \subseteq V$,
we will use the notation $G - S$ to refer to the graph obtained by removing all
vertices from $V$ that are in $S$, and all edges from $E$ with an endpoint in $S$.

\begin{lemma}
  Suppose $G = (L, R, E)$ is a bipartite graph with a unique perfect matching. Then
  there is at least one vertex in $L$ with degree 1.
\end{lemma}

\begin{proof}
  Let $M$ be the unique perfect matching in $G$. We will prove this lemma
  by contradiction. Suppose every vertex in L had at least
  degree 2.\footnote{
      Since there is a perfect matching every vertex must have non-zero degree,
      so if there is no vertex of degree 1 then every vertex must have degree
      at least 2.}
  Let us pick an arbitrary vertex $L$, and call this vertex $u_0$.
  Suppose we greedily construct a path $P$ starting at $u_0$, by successively picking
  arbitrary unmatched edges from $L$ to $R$, and matched edges from $R$ to $L$, until
  we cannot extend the path any further. Let $P = (u_0,\ldots,u_k)$.
  Since we have assumed that every vertex $u \in L$ has degree at least 2, each $u$
  must have at least one unmatched edge to a vertex in $R$, and since $M$ is a
  perfect matching, each vertex $v \in R$ must have a matched edge to a vertex in $L$.
  Therefore if $u_k \in L$ there is an unmatched edge
  $(u_k, u_i)$ for some $i < k$, and if $u_k \in R$ there is a matched edge
  $(u_k, u_i)$ for some $i < k$ (otherwise we could extend the path further).
  In either case, $(u_i,\ldots,u_k)$ is an alternating cycle in $G$ with respect
  to $M$. Augmenting $M$ with respect to this cycle gives a new maximum matching,
  contradicting that $M$ was unique.
\end{proof}

\begin{lemma}
  Suppose $G = (L, R, E)$ is a bipartite graph with a unique perfect matching. Let
  $u$ be a vertex in $L$ with degree 1, and let $v$ be its neighbor in $R$. Then
  the graph $G' = G - \{u, v\}$ is also a bipartite graph
  with a unique perfect matching.
\end{lemma}

\begin{proof}
  The fact that there is such a vertex $u$ in $L$ follows from the previous lemma.
  Since $u$ has degree 1, any perfect matching $M$ must involve $u$ being matched
  to $v$. In other words, any perfect matching $M$ in $G$ must correspond exactly to
  a perfect matching $M - \{(u, v)\}$ in $G'$, and vice versa. Thus, since $G$ has a
  unique perfect matching, $G'$ must also have one.
\end{proof}

These lemmas motivate a simple iterative algorithm for constructing our desired
functions $g$ and $h$, as follows:

\begin{algorithm}
  \caption{Map to half-graph}
  \begin{algorithmic}[1]
    \Procedure{ConstructMap}{$G$}
    \State $g\gets$ \Call{CreateEmptyMap}{}
    \State $h\gets$ \Call{CreateEmptyMap}{}
    \For{$i \in \{n,\ldots,1\}$}
      \State $u\gets$ \Call{GetMinV}{$L$}\Comment{Get a vertex with min. degree in $L$}
      \State $v\gets$ \Call{GetNeighbor}{$G, u$}\Comment{Get a neighbor of $u$ in $G$}
      \State $g(u)\gets u_i$
      \State $h(v)\gets v_i$
      \State $G\gets G - \{u, v\}$
    \EndFor
    \State \textbf{return} $g, h$
    \EndProcedure
  \end{algorithmic}
\end{algorithm}

First of all, the above two lemmas imply that at the start of each iteration
$G$ has a unique perfect matching, and that there is a vertex with degree 1.
Therefore lines 5 and 6 will always succeed in running.
In addition, since every vertex in the subgraph gets assigned to one vertex in $G$,
which is subsequently removed from $G$,
it is clear that this algorithm defines a one-to-one correspondence
between the vertices in $G$ and the vertices of the half-graph.
It only remains to be shown that the algorithm is polynomial time, and that
for every $(u, v) \in E$ that $(g(u), h(v))$ is an edge of the half-graph.

The proof that the algorithm is polynomial time is trivial: each iteration
involves finding a vertex of minimum degree from $L$, picking a neighbor
of this vertex, making some assignments, and removing some vertices and
edges from the graph. Each of these parts is at worst $O(n + m)$ depending on
the data structure used to represent $G$, and the loop is run $O(n)$ times,
so the complexity of the algorithm is at worst $O(n^2 + nm)$.

Now we show that the edges in $G$ are mapped by $g$ and $h$ to edges
in the half-graph. Consider an arbitrary edge
$(u, v) \in G$, where $u \in L$ and
$v \in R$. Let $k$ be the value of the loop variable $i$ at the iteration
where $u$ is removed from the graph. By construction this means that
$g(u) = u_k$.
Since $G$ has a minimum vertex degree of 1 at the start of each iteration,
$u$ must have had a single neighbor in $R$ at the iteration where it was
removed, which was mapped by $h$ to $v_k$.
This means all other possible neighbors of $u$ must have
been removed at earlier iterations of the algorithm, and therefore each
mapped to $v_j$ for some $j > k$. In either case we have $h(v) = v_j$,
for some $j \ge k$. Thus $(g(u), h(v)) = (u_k, v_j)$ is an edge
in the half-graph, as required.

\end{enumerate}
\end{solution}
\begin{problem}[2]
Define a fractional perfect matching in a (not necessarily bipartite) graph $G = (V, E)$ to be a matrix
$X = \left (x_{uv} \right)$, with rows and columns indexed by elements of V , that satisfies:
\begin{enumerate}
\item degree constraints: $\sum_{v\in V}x_{uv} = 1$ for all $u \in V$.
\item odd set constraints: $\sum_{u\in S} \sum_{v \not \in S} x_{uv} \geq 1$ for every set $S$ of odd cardinality.
\item symmetry: $x_{uv} = x_{vu}$ for all $u, v\in V$
\item non-negativity: $x_{uv} \geq 0$ for all $u, v \in V$.
\item supported on edges: $x_{uv} = 0$ if $(u, v) \not \in E$.
\end{enumerate}

(a) If G is a bipartite graph with bipartition $V = L \cup R$, prove that the odd set constraints
are implied by the other constraints, and that a matrix $X$ satisfies the definition of fractional perfect
matching given above, if and only if it has the block form
\[ X = \begin{pmatrix}0 & Y \\ Y^T & 0 \end{pmatrix},\]
where $Y$ , the submatrix of entries $x_{uv}$ with $(u, v) \in L \times R$, satisfies the definition of fractional perfect
matching given in class and in the lecture notes. (Thus, in some sense, the definition of fractionalperfect matchings in this exercise extends the definition for bipartite graphs given in class to the case
of non-bipartite graphs.)

(b*) For a fractional perfect matching $X$, define its support, Supp$(X)$, to be the set of edges $(u, v)$
with $x_{uv} > 0$. If $G$ is a graph with $n$ vertices, $X$ is a fractional perfect matching, and $M \subset \text{Supp}(X)$ is
a matching with strictly fewer than $n/2$ edges, prove that there is an $M$-augmenting path contained in
Supp$(X)$. For half credit, prove it in the special case that $G$ is bipartite.

(c) Every perfect matching $M$ in $G$ gives rise to a matrix $X(M)$, by setting $x_{uv} = 1$ if edge $(u, v)$
belongs to $M$ , and $x_{uv} = 0$ otherwise. Call such a matrix a perfect matching matrix. Assuming $G$ is
bipartite, give an algorithm that takes a matrix $X$ that is a fractional perfect matching, and outputs
a representation of $X$ as a convex combination of perfect matching matrices. In other words, your
algorithm’s output should be a list of matchings $M_i$ , and coefficients $c_i \geq 0$, such that $X = \sum_i c_i X(M_i)$.
You may make use of part (2b) in designing and analyzing your algorithm, even if you didn’t solve (2b).
Remark: The fact that a fractional perfect matching matrix can be expressed as a convex combination
of perfect matching matrices, for bipartite $G$, is the Birkhoff-von Neumann Theorem. The fact is still
true for non-bipartite $G$, but it is significantly harder to prove, and it is not part of this assignment.
\end{problem}



\begin{solution}
\begin{enumerate}[label=(\Alph*)]
\item
First, we show that the odd set constraints are implied by the other constraints. Let $S$ be an odd set and let $S_L = S \cap L$ and $S_R = S \cap R$. Since $S$ is odd cardinality, $S_L$ and $S_R$ must have different cardinalities; without loss of generality, let $S_L$ be bigger, that is, $|S_L| - |S_R| \geq 1$. Then we have
\begin{align*}
\sum_{\substack{v \in S \\ w \not \in S}} x_{vw} &= \sum_{\substack{v \in S_L \\ w \not \in S}} x_{vw} + \sum_{\substack{v \in S_R \\ w \not \in S}} x_{vw} \text{\quad(Because $S_L,S_R$ is a partition)} \\
&= \left (\sum_{\substack{v \in S_L \\ w \not \in S_L}} x_{vw} - \sum_{\substack{v \in S_L \\ w \in S_R}} x_{vw}\right ) + \left (\sum_{\substack{v \in S_R \\ w \not \in S_R}} x_{vw} - \sum_{\substack{v \in S_R \\ w \in S_L}} x_{vw}\right ) \text{\quad (By conditioning on $w\in S$)} \\
&= \sum_{\substack{v \in S_L \\ w \not \in S_L}} x_{vw} + \sum_{\substack{v \in S_R \\ w \not \in S_R}} x_{vw} - 2\sum_{\substack{v \in S_R \\ w \in S_L}} x_{vw} \text{\quad (By reflexivity of $X$)}\\
&\geq \sum_{\substack{v \in S_L \\ w \not \in S_L}} x_{vw} + \sum_{\substack{v \in S_R \\ w \not \in S_R}} x_{vw} - 2\sum_{\substack{v \in S_R \\ w \not \in S_R}} x_{vw} \text {\quad (By nonnegativity of $X$)}\\
& = |S_L| + |S_R| - 2|S_R| = |S_L|-|S_R| \text{\quad (By degree constraint and bipartiteness)}\\
& \geq 1 \text {\quad (By cardinality argument above)}
\end{align*}

Next, we show that if $X$ satisfies the above if and only if it is of the form 
\[ X = \begin{pmatrix} 0 & Y \\ Y^T & 0\end{pmatrix} \]
where $Y$ is nonnegative and doubly stochastic.

The forward direction: if $X$ satisfies the above, it is certainly of block form
\[\begin{pmatrix} 0 & Y \\ Z & 0\end{pmatrix}\]
due to bipartiteness. The symmetry condition shows $Z=Y^T$. The degree constraints show that $Y$ and $Z$ are row-stochastic, which together show that $Y$ is doubly stochastic. Finally, non-negativity of $X$ and $0$ show non-negativity of $Y$.

The reverse direction: suppose $Y$ is nonnegative and doubly stochastic. Then trivially the resulting $X$ is also nonnegative and doubly stochastic by construction, and hence row-stochastic. This gives us the non-negativity and degree constraints. By construction, $X$ is symmetric, which gives us the symmetry constraint. The supported-on-edges constraint is given by the observation that $X$ is exactly the adjacency matrix of a bipartite graph $G$ whose edge weights are given by the matrix $Y$.

\item
  
%%% SOLUTION TO 2(B)
  
First, we will define a \emph{fractional superperfect matching}, which is a
matrix $X=(x_{uv})$ with the same constraints as a fractional perfect matching,
except that we relax the degree constraint to
\[\sum_{v\in V} x_{uv} \geq 1\]
A vertex $v$ where $\sum x_{uv} > 1$ we will call a \emph{supervertex}.

Second, we will define a \emph{search tree} $T$ in a graph $G$ with a fractional
superperfect matching $X$ and with respect to a matching $M$ in $\text{Supp}(X)$
as a rooted tree, subgraph of $\text{Supp}(X)$ where:

\begin{itemize}
\item The root, $v$ of $T$ is free in $M$
\item All other vertices $m$ in $T$ are matched
\item Every path from the root $v$ to any leaf has an odd number of vertices
      and is $M$-alternating
\item Vertices an even length away from the $v$ (including $v$ itself) are
      called \emph{black} and an odd length away are called \emph{white}
\item All supervertices in $G$ are in the tree, and are black vertices.
\end{itemize}

We will show that, given a fractional superperfect matching $X$ and a graph $G$
and matching $M$ using strictly fewer than $n/2$ edges and a tree $T$, we can always find
one of:

\begin{itemize} 
\item an augmenting path,
\item a larger search tree $T'$, or
\item a flower in $T$ such that collapsing its blossom $B$ gives a new,
      smaller $G/B$ with a fractional superperfect matching $X/B$ and a new search tree $T/B$.
\end{itemize}

Since other than the root $v$, every vertex in $T$ also has its match in $T$,
$|T|$ is odd.
Therefore by the odd set constraint, there is a supported edge out
of $T$. Consider all such edges. Then there are 3 cases:

\begin{itemize}
\item There is a supported edge from a black vertex $b$ to a free vertex $w$.
      Then the path from $v$ to $b$ to $w$ is an augmenting path ($v$ to $b$
      is possibly empty).
\item There is a supported edge from a black vertex $b$ to a matched vertex $m$.
      Then a new search tree $T$ can be created by adding the edges $(b,m)$ and
      $(m, m')$ where $m'$ is matched to $m$. Note that by definition of a serach
      tree, neither $m$ nor $m'$ is a supervertex.
\item All supported edges from $T$ to the rest of the graph are from white vertices.
\end{itemize}

In this last case we will find a blossom. First we will argue that there is a
supported edge between two black vertices. Note that all outgoing edges from
black vertices go to either white or black vertices, and that all
supervertices are black. So, suppose there were no black-to-black edges; then we have
\[|W| > \sum_{\substack{(b,w)\\ b \in B, w\in W}} x_{bw} \geq |B|\]
where $W$ are the white vertices and $B$ are the black vertices, by degree constraints
(the first equality is strict because $W$ has some edges outside of $T$, and the
second inequality follows because $B$ may have supervertices). But $|B| = |W| + 1$.

Since we have a black-to-black edge, we can find a blossom. Call the edge $(b_1, b_2)$.
Consider the unique simple paths $P_i$, where $P_i$ is the path from the root to $b_i$.
Let $(u_0,\ldots,u_k)$ be the longest common prefix of $P_1$ and $P_2$. We will argue
that these paths define a flower with $u_k$ as the fulcrum, and and the suffixes of
the paths along with $u_k$ defining the blossom. Since white to black edges in the tree
are by construction always matched, there cannot be two separate edges from a white
edge away from the root. Therefore $u_k$ must be black. Furthermore since the paths
end at black vertices, the two suffixes together contain an even number of vertices.
Finally, $(b_1, b_2)$ must be unmatched as at least one of these vertices is
already matched to a separate white vertex (both cannot be the root). Therefore
along with the fulcrum these two suffixes joined at their ends by $(b_1, b_2)$
must define a blossom. 

Let $L$ be the blossom. We will now argue that the smaller graph $G/L$ by collapsing
the blossom to a single vertex has a fractional superperfect $X/L$ (obtained by collapsing
the rows/columns corresponding to $L$ to a single row/column by summing over them)
matching with a matching $M/L$ (obtained by removing edges in $L$)
and new search tree $T/L$ (obtained by collapsing the blossom to a single vertex in $T$).

To show the new $X/L$ is superperfect, we must show all five conditions hold. Symmetry,
non-negativity are trivial. In the case of supported edges it follows from the fact
that the row/column corresponding to $L$ is formed by summation, so if there were no
edges from a vertex $u$ to any vertex in $L$ then all the prior entries were zero
(previously supported on edges), so the new corresponding entry remains zero.
To show the odd set constraint,
let $S$ be an odd set in $G/L$. If it only contains vertices that also existed in
$G$, then the odd set constraint on $X$ implies it for $X/L$. If the odd set $S$
contains the $L$-vertex, then we could form a new set $S' \subset V(G)$ by
expanding the collapsed $L$ vertex to all the vertices in $L$. This is an odd set
in $G$ so the sum of the weights out is at least 1, and by construction of $X/L$ the
sum out of the set is preserved.

To show that the new $T/L$ is still a search tree with respect to $M/L$, note that
all subtrees in $T\\L$ rooted at a vertex in $L$ must have a white
root (since by construction the parent of every black vertex excluding the root
is a matched white vertex, but all white vertices in the blossom are matched
to a black vertex in the blossom). Therefore since the collapsed blossom vertex
in $T/L$ is black, we do not need to change the colors of its descendants, so
any prior supervertex remains black.

Note as well that since there are fewer than $n/2$ edges in the original matching
at least one vertex must be free, so it is always possible to choose an initial root
for the tree.
This is a valid search tree at this time because there are no supervertices
at this time (because the original fractional perfect matching is perfect).
Therefore we can always start off with a valid search tree.
Since we can only collapse the tree and associated superperfect matching a finite 
number of times, we must be able to find an augmenting path eventually.
This path is augmenting with respect to the collapsed matching in the
collapsed graph after a finite number of bollosom collapses.
However as proven in lectures $G$ has an augmenting path
if and only if $G/L$ has an augmenting path, so therefore the original graph
$G$ must have an augmenting path.

\item

\begin{lemma}
  Suppose $X$ a fractional perfect matching for a bipartite graph $G$,
  $M$ is a perfect matching in $G$ contained in the support of $X$,
  $X^M$ is the integer fractional perfect matching corresponding to $M$,
  $0 < \alpha \leq 1$, and $\beta$ is equal to the minimum weight in $\alpha X$
  of the edges in $M$. Then $\alpha X - \beta X^M = \gamma X^*$ for some
  $0 < \gamma < 1$ and fractional perfect matching $X^*$, unless $X = X^M$.
\end{lemma}

\begin{proof}

  We need to argue that there exists a $X^*$ satisfying the fractional
  perfect matching and corresponding $\gamma$.
  Firstly, we can observe that since $\beta$ is the minimum element of
  $\alpha X$ contained in $M$, $\alpha \geq \beta$. However $\alpha = \beta$
  if and only if the minimum weight of the matched edges in $X$ is 1, meaning
  that $X = X^M$. But are only required to construct $X^*$ and $\gamma$ if
  this is not true, so henceforth we will assume instead that $\alpha > \beta$.

  The degree constraints require that every row and column in $X^*$ add up to
  1. Since $X$ and $X^M$ satisfy these constraints, it
  must be the case that every row and column in $\alpha X - \beta X^M$
  add up to $\alpha - \beta$. Therefore if we set $\gamma = \alpha - \beta$,
  then $X^* = \frac{\alpha}{\gamma} X - \frac{\beta}{\gamma} X^M$ must
  also satisfy the degree constraints. Since $1 \geq \alpha > \beta > 0$, it must
  be the case that $\gamma$ defined this way is between 0 and 1.
  
  We will finally argue that $\gamma$ and $X^*$ defined this way satisfy
  the remaining fractional perfect matching constraints. First of all, since
  $X$ and $X^M$ are both symmetrical matrices, $X^*$ is trivially symmetrical
  as well so the symmetry constraints are satisfied. Secondly, since $X$
  and $X^M$ satisfy non-negativity, the support of $X^M$ is a subset of
  the support of $X$ by construction, and $\beta$ is defiend as the minimum
  element in $\alpha X$ overlapping with the support of $X^M$, it must
  be the case that every entry in
  $\alpha X - \beta X^M$ is non-negative. Therefore $X^*$ must satisfy
  the non-negativity constraints since
  $\gamma > 0$. Thirdly, for every $(u, v) \notin E$, $X_{uv} = X^M_{uv} = 0$
  since $X$ and $X^M$ satisfy the supported on edges constraints.
  Therefore $X^*_{uv} = 0$, so $X^*$ is also supported on edges. Finally, given
  the result of 2(a), since $G$ is bipartite and all other constraints
  are satisfied, $X^*$ must also satisfy the odd-set constraints.
  Therefore since all constraints are satisfied, $X^*$ must be a fractional
  perfect matching.
\end{proof}

Given this lemma and the result of 2(b), a polynomial time algorithm for
converting a fractional perfect matching $X$ to a convect combination of perfect
matchings, follows naturally, as follows below.

\begin{algorithm}
  \caption{Construct perfect matching convex combination}
  \begin{algorithmic}[1]
    \Procedure{ConstructConvexCombination}{$X$}
    \State $i\gets 0$
    \While{true}
    \State $M\gets$ \Call{BuildPerfectMatching}{$X$}
    \State $X_i\gets$ \Call{MatchingToMatrix}{$M$}
      \State $(u, v)\gets$ \Call{MinWeightedMatchedEdge}{$X,M$}
      \State $\lambda_i\gets X_{uv}$
      \State $X\gets X - \lambda_i X_i$
      \If{$X = 0$}\Comment{If all entries in X are zero}
        \State \textbf{break}
      \EndIf
      \State $i\gets i + 1$
    \EndWhile
    \State \textbf{return} $\{(\lambda_0, X_0),\ldots,(\lambda_i, X_i)\}$
    \EndProcedure
  \end{algorithmic}
\end{algorithm}

To argue the correctness of this algorithm, we can first observe that as a loop
invariant at the start of each iteration of the while loop $X = \alpha X'$ for
some fractional perfect matching $X'$ and some $0 < \alpha \leq 1$. This
is obviously true at the start of the first iteration with $\alpha = 1$. To
see that it is true in subsequent iterations note that the subtraction performed
on line 8 corresponds to the form $\alpha X - \beta X^M$ appearing in the above
lemma, with each variable satisfying the conditions in the lemma. In addition, as soon
as $X = X_i$ (corresponding to $X = X^M$ in the lemma) we break out
of the loop, meaning that this could never have been true before the start
of an iteration. Therefore the loop invariant follows from the lemma.

Secondly, we argue that the perfect matching $M$ from the support of $X$
on line 4 can always be constructed. Since $X = \alpha X'$
for some fractional perfect matching $X'$ with $\alpha > 0$, the support
of $X$ is the same as the support of $X'$, and is non-zero.
The result of 2(b) says that if we have a non-perfect
matching contained in the support of $X$ an augmenting path with respect to
that mataching can be found. This means any non-perfect matching in the support
of $X$ is non-maximal, so there is a perfect matching. Thus $M$ can be constructed
using any subroutine for constructing a maximal matching in a bipartite graph
(e.g. Hopcroft-Karp algorithm).

Thirdly, we argue that the algorithm always terminates. Each iteration of the
algorithm involves setting $\lambda_i$ to the value of the minimum matched
edge in $X$. Therefore subtracting $X$ by $\lambda_i X_i$ must strictly reduce
the size of the support of $X$. Therefore there can only be a finite number
of iterations before $X$ becomes 0.

Fourthly, we argue that the algorithm returns a convex combination of perfect
matchings adding up to $X$. Clearly by construction $X = \sum_{i = 0}^n{\lambda_i X_i}$
where $n$ is the number of weight-matching pairs returned by the algorithm.
This is because the algorithm terminates when $X$ is exactly equal to 0.
In addition, because at each iteration $X = \alpha X'$, and each weight is
defined as a value in $X$, each weight must satisfy $0 < \lambda \leq 1$.
Also, by construction, each $X_i$ is a perfect matching matrix.
Finally, because each row and column of the original $X$ sums to 1,
and each row of column of each $X_i$ sum to 1, given that
$X = \sum_{i = 0}^n{\lambda_i X_i}$ it must be the case that the weights
sum to 1. Therefore these weights and matrices returned by the algorithm
represent $X$ as a convex combination of perfect matchings, as required.

Finally, we argue that this algorithm is polynomial time. Since each loop
of the algorithm reduces the size of the support of $X$, there can be at
most $m$ iterations. Every operation in the loop except for constructing
the perfect matching is obviously at worst $O(n^2)$, since at worst they
involve iterating through every element of $X$. Let us say for the sake of
argument that we use the Hopcroft-Karp algorithm for constructing the
perfect matching $M$, which has complexity $O(m\sqrt{n})$. Then the
complexity of our entire algorithm would be $O(mn^2 + m^2\sqrt{n})$.

\end{enumerate}
\end{solution}
\begin{problem}[3]
(Problem 3)
\end{problem}
\begin{solution}
(We are just showing 3b, from which 3a follows trivially)

First, we will show that player 2 has a winning strategy if the graph has a perfect matching. Their strategy is: whenever player 1 chooses a vertex $v$, choose its match, $v'$. This strategy maintains the invariant that whenever player 1 choose a vertex, its match has not been chosen, so $v'$ is always a legal play. Since player 2 always has a legal play but the graph is finite, player 1 must eventually not have a legal play.

Second, we will show that player 1 has a winning strategy if the graph lacks a perfect matching. They choose any maximum matching $M$ and choose a free vertex $v$, and they play $v$. Player 1's strategy is: whenever player 2 chooses a vertex $m$, take its match $m'$. This strategy maintains the same invariant as above, but it remains to show that this is always possible -- that player 2 must always choose a matched vertex. Suppose at some point player 2 choose a free vertex $w$. This means that the path of choices so far $(v,...,w)$ is augmenting for $M$. But, since $M$ is maximum this is a contradiction. Since player 1 always has a legal play but the graph is finite, player 2 must eventually not have a legal play.

Since whether or not a graph has a perfect matching is decidable in polynomial time (as shown in lecture), this gives a polynomial-time decider for who has the winning strategy.

\end{solution}
\end{document}


