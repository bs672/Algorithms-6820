\documentclass{hmcpset}
\usepackage{amsmath}
\name{Andrew Bennett, Richard Strong Bowen}
\class{CS6820}
\assignment{Problem Set 1}
\duedate{15 September 2017}
\begin{document}
\begin{problem}[1]
The half-graph is a bipartite graph with vertex sets $L_0 = \{u_1,\ldots,u_n\}$ and $R_0 = \{v_1,\ldots,v_n\}$ and edge set $E = \{(u_i, v_j) | i \leq j\}$.
\begin{enumerate}
\item Prove that the half-graph has a unique perfect matching.
\item Let $G = (L,R,E)$ be a bipartite graph with $|L| = |R| = n$. Suppose $G$ has a unique perfect matching. Prove that $G$ is isomorphic to a subgraph of the half-graph. Your proof should be polynomial-time constructive.
\item If $\pi$ is a permutation of ${1,\ldots,n},$, let $H_\pi$ denote the "permuted half graph" with vertex sets $L_0$ and $R_0$ as in part (a) and with edge set $E_\pi = \{(u_i, v_{\pi(j)} | i \leq j\}$. If ALG is any algorithm for the online matching problem -- potentially randomized -- let $M_\pi$ denote the (potentially random) matching that results from running ALG on the input instance $H_\pi$ with the vertices of $L_0$ arriving in the order $u_1, u_2, \ldots, u_n$. Prove that the expected cardinality of $M_\pi$ is at most $(1-\frac{1}{e} + o(1))n$, where the $o(1)$ term tends to zero as $n\rightarrow \infty$.
\end{enumerate}
\end{problem}
\begin{solution}
(Solution 1 stub)
\begin{enumerate}
\item (a)

First it is obvious that there exists a perfect matching for the half graph, since
$M = \{(u_1,v_1),\ldots,(u_n,v_n)\}$ is a matching containing every vertex in $G$.
It only remains to be shown that this matching is unique.

The uniqueness of the matching can be shown via induction. Let $P(n)$ be the
proposition that the half-graph of size $n$ has a unique perfect matching,
as described above. $P(1)$ is clearly true, because with $n = 1$ there are only
two vertices with a single edge, so uniqueness of the perfect matching here
is trivial.

Now suppose that proposition $P(k - 1)$ is true. Consider the half graph $G_k$
containing $2k$ vertices. By construction $u_k$ has a single neighbor in $G_k$,
which is $v_k$. This means any perfect matching must involve $u_k$ and $v_k$
being matched together. Therefore $G_k$ has a unique perfect matching if and
only if the remaining graph after removing $u_k$, $v_k$, and any adjacent edges
has a unique perfect matching. Let us call this smaller graph containing
$2k -2$ vertices $H_k$.

We can observe that $H_k$ is identical to the half graph of size $k - 1$. This
is because it contains vertices $\{u_1,\ldots,u_{k-1},v_1,\ldots,v_{k-1}\}$ with
edges satisfying the same equations as for the half graph (the edge set is
$E = \{(u_i, v_j) | i \leq j\}$). Therefore we know $H_k$ has a unique perfect
matching given proposition $P(k - 1)$ which we have assumed to be true.
So $P(k - 1) \implies P(k)$.

We have proven the base case and the inductive case, so we have our desired
result that the half graph has a unique perfect matching for every $n \in \mathbb{N}$.


\item (b)
\end{enumerate}
\end{solution}
\begin{problem}[2]
(Problem 2 stub)
\end{problem}
\begin{solution}
(Solution 2 stub)
\begin{enumerate}
\item[a] 
First, we show that the odd set constraints are implied by the other constraints. Let $S$ be an odd set and let $S_L = S \cap L$ and $S_R = S \cap R$. Since $S$ is odd cardinality, $S_L$ and $S_R$ must have different cardinalities; without loss of generality, let $|S_L| - |S_R| \geq 1$. Then we have
\begin{align*}
\sum_{\substack{v \in S \\ w \not \in S}} x_{uv} &= \sum_{\substack{v \in S_L \\ w \not \in S}} x_{uv} + \sum_{\substack{v \in S_R \\ w \not \in S}} x_{uv} \text{\quad(Because $S_L,S_R$ is a partition)} \\
&= \left (\sum_{\substack{v \in S_L \\ w \not \in S_L}} x_{uv} - \sum_{\substack{v \in S_L \\ w \in S_R}} x_{uv}\right ) + \left (\sum_{\substack{v \in S_R \\ w \not \in S_R}} x_{uv} - \sum_{\substack{v \in S \\ w \not \in S}} x_{uv}\right ) \text{\quad (By conditioning on $w\in S$)} \\
&= \sum_{\substack{v \in S_L \\ w \not \in S_L}} x_{uv} + \sum_{\substack{v \in S_R \\ w \not \in S_R}} x_{uv} - 2\sum_{\substack{v \in S_R \\ w \in S_L}} x_{uv} \text{\quad (By reflexivity of $X$)}\\
&\leq \sum_{\substack{v \in S_L \\ w \not \in S_L}} x_{uv} + \sum_{\substack{v \in S_R \\ w \not \in S_R}} x_{uv} - 2\sum_{\substack{v \in S_R \\ w \not \in S_R}} x_{uv} \text {\quad (By nonnegativity of $X$)}\\
& = |S_L| + |S_R| - 2|S_R| = |S_L|-|S_R| \text{\quad (By degree constraint and bipartiteness)}\\
& \geq 1 \text {\quad (By cardinality argument above)}
\end{align*}
\item[b] First, we will define a \emph{fractional superperfect matching}, which is a matrix $X=(x_{uv})$ with the same constraints as a fractional perfect matching, except that we relax the degree constraint to
\[\sum_{v\in V} x_{uv} \geq 1\]
A vertex $v$ where $\sum x_{uv} > 1$ we will call a \emph{supervertex}.

Second, we will define a \emph{search tree} $T$ in a graph $G$ with a fractional superperfect matching $X$ and with respect to a matching $M$ in $\text{Supp}(X)$ as a rooted tree, subgraph of $\text{Supp}(X)$ where:
\begin{itemize}
\item The root, $v$ of $T$ is free in $M$
\item All other vertices $m$ in $T$ are matched
\item Every path from the root $v$ to any leaf has an odd number of vertices and is $M$-alternating
\item Vertices an even length away from the $v$ (including $v$ itself) are called \emph{black} and an odd length away are called \emph{white}
\item All supervertices in $G$ are in the tree, and are black vertices.
\end{itemize}

We will show that, given a fractional superperfect matching $X$ and a graph $G$ and matching $M$ using at most $n/2$ edges and a tree $T$, we can always find one of:
\begin{itemize} 
\item an augmenting path,
\item a larger search tree $T'$, or
\item a flower in $T$ such that collapsing its blossom $B$ gives a new, smaller $G/B$ with a fractional superperfect matching $X/B$ and a new search tree $T/B$.
\end{itemize}

Since other than the root $v$, every vertex in $T$ also has its match in $T$, $|T|$ is odd. Therefore by the odd set constraint, there is a supported edge out of $T$. Consider all such edges. Then there are 3 cases:
\begin{itemize}
\item There is a supported edge from a black vertex $b$ to a free vertex $w$. Then the path from $v$ to $b$ to $w$ is an augmenting path ($v$ to $b$ is possibly empty).
\item There is a supported edge from a black vertex $b$ to a matched vertex $m$. Then a new search tree $T$ can be created by adding the edges $(b,m)$ and $(m, m')$ where $m'$ is matched to $m$. Note that by definition of a serach tree, neither $m$ nor $m'$ is a supervertex.
\item All supported edges from $T$ to the rest of the graph are from white vertices.
\end{itemize}
In this last case we will find a blossom. First we will argue that there is a supported edge between two black vertices. Note that all outgoing edges from black vertices go to either white or black vertices, and that all supervertices are black. So, suppose there were no black-to-black edges; then we have
\[|W| > \sum_{(b,w): \text{$b$ is black and $w$ is white}} x_{bw} \geq |B|\]
by degree constraints (the first equality is strict because $W$ has some edges outside of $T$, and the second inequality follows because $B$ may have supervertices). But $|B| = |W| + 1$.

Since we have a black-to-black edge, we can find a blossom. Call the edge $(b_1, b_2)$. Consider the unique simple paths $P_i$, where $P_i$ is the path from the root to $b_i$. Here there are two cases: one of $P_i$ completely contains the other, or not. In first case, without loss of generality say $P_2$ contains $P_1$. Then the path $(P_2, (b_2,b_1))$ is a blossom. Otherwise, suppose that the paths are the same up some vertex $u$. Then the path $(P_1, (b_1, b_2), P'_2)$ is a blossom, where $P'_2$ is the reverse of $P_2$ and stopping at $u$ rather than $v$.

All that remains to be shown is that the new small $G/B$ has a fractional superperfect matching with a matching $M/B$ and new search tree $T/B$.

To show the new $X/B$ is superperfect, we must show all five conditions hold. Symmetry, non-negativity, and supported-on-edges are trivial. To show the odd set constraint, let $S$ be an odd set in $G/B$. If it only contains vertices that also existed in $G$, then the odd set constraint on $X$ implies it for $X/B$. If the odd set $S$ contains the $B$-vertex, note that $S$ minus the $B$-vertex plus all the vertices in $B$ is an odd set in $G$, so the odd-set constraint in $X$ gives it to us in $X/B$. Finally, the degree constraint needs to be checked on the new $B$ vertex. It is just implied by the odd set constraint on $X$.

To show that the new $T/B$ is still a search tree with respect to $M/B$, note that by construction, all outgoing edges from $B$ to $T$ are unmatched. So the $B$ vertex will still live on the black side of $T/B$, perhaps taking the role of $v$ if the flower had no stem, and all other supervertices in $T/B$ are also still on the right side.

The obvious inductive argument then shows that we can find an augmenting path, and note that for a fractional perfect matching (not superperfect) $X$ with supported matching $M$, a search tree can be constructed simply by choosing any free vertex to be $T$. This proves the theorem.

\item (c)
\end{enumerate}
\end{solution}
\begin{problem}[3]
(Problem 3 stub)
\end{problem}
\begin{solution}
(We are just showing 3b, from which 3a follows trivially)

First, we will show that player 2 has a winning strategy if the graph has a perfect matching. Their strategy is: whenever player 1 chooses a vertex $v$, choose its match, $v'$. This strategy maintains the invariant that whenever player 1 choose a vertex, its match has not been chosen, so $v'$ is always a legal play. Since player 2 always has a legal play but the graph is finite, player 1 must eventually not have a legal play.

Second, we will show that player 1 has a winning strategy if the graph lacks a perfect matching. They choose any maximum matching $M$ and choose a free vertex $v$, and they play $v$. Player 1's strategy is: whenever player 2 chooses a vertex $m$, take its match $m'$. This strategy maintains the same invariant as above, but it remains to show that this is always possible -- that player 2 must always choose a matched vertex. Suppose at some point player 2 choose a free vertex $w$. This means that the path of choices so far $(v,...,w)$ is augmenting for $M$. But, since $M$ is maximum this is a contradiction. Since player 1 always has a legal play but the graph is finite, player 2 must eventually not have a legal play.

Since whether or not a graph has a perfect matching is decidable in polynomial time (as shown in lecture), this gives a polynomial-time decider for who has the winning strategy.

\end{solution}
\end{document}


