\documentclass{hmcpset}
\usepackage{amsmath}
\name{Andrew Bennett, Richard Strong Bowen}
\class{CS6820}
\assignment{Problem Set 1}
\duedate{15 September 2017}
\begin{document}
\begin{problem}[1]
The half-graph is a bipartite graph with vertex sets $L_0 = \{u_1,\ldots,u_n\}$ and $R_0 = \{v_1,\ldots,v_n\}$ and edge set $E = \{(u_i, v_j) | i \leq j\}$.
\begin{enumerate}
\item Prove that the half-graph has a unique perfect matching.
\item Let $G = (L,R,E)$ be a bipartite graph with $|L| = |R| = n$. Suppose $G$ has a unique perfect matching. Prove that $G$ is isomorphic to a subgraph of the half-graph. Your proof should be polynomial-time constructive.
\item If $\pi$ is a permutation of ${1,\ldots,n},$, let $H_\pi$ denote the "permuted half graph" with vertex sets $L_0$ and $R_0$ as in part (a) and with edge set $E_\pi = \{(u_i, v_{\pi(j)} | i \leq j\}$. If ALG is any algorithm for the online matching problem -- potentially randomized -- let $M_\pi$ denote the (potentially random) matching that results from running ALG on the input instance $H_\pi$ with the vertices of $L_0$ arriving in the order $u_1, u_2, \ldots, u_n$. Prove that the expected cardinality of $M_\pi$ is at most $(1-\frac{1}{e} + o(1))n$, where the $o(1)$ term tends to zero as $n\rightarrow \infty$.
\end{enumerate}
\end{problem}
\begin{solution}
(Solution 1 stub)
\begin{enumerate}
\item (a)

First it is obvious that there exists a perfect matching for the half graph, since
$M = \{(u_1,v_1),\ldots,(u_n,v_n)\}$ is a matching containing every vertex in $G$.
It only remains to be shown that this matching is unique.

The uniqueness of the matching can be shown via induction. Let $P(n)$ be the
proposition that the half-graph of size $n$ has a unique perfect matching,
as described above. $P(1)$ is clearly true, because with $n = 1$ there are only
two vertices with a single edge, so uniqueness of the perfect matching here
is trivial.

Now suppose that proposition $P(k - 1)$ is true. Consider the half graph $G_k$
containing $2k$ vertices. By construction $u_k$ has a single neighbor in $G_k$,
which is $v_k$. This means any perfect matching must involve $u_k$ and $v_k$
being matched together. Therefore $G_k$ has a unique perfect matching if and
only if the remaining graph after removing $u_k$, $v_k$, and any adjacent edges
has a unique perfect matching. Let us call this smaller graph containing
$2k -2$ vertices $H_k$.

We can observe that $H_k$ is identical to the half graph of size $k - 1$. This
is because it contains vertices $\{u_1,\ldots,u_{k-1},v_1,\ldots,v_{k-1}\}$ with
edges satisfying the same equations as for the half graph (the edge set is
$E = \{(u_i, v_j) | i \leq j\}$). Therefore we know $H_k$ has a unique perfect
matching given proposition $P(k - 1)$ which we have assumed to be true.
So $P(k - 1) \implies P(k)$.

We have proven the base case and the inductive case, so we have our desired
result that the half graph has a unique perfect matching for every $n \in \mathbb{N}$.


\item (b)
\end{enumerate}
\end{solution}
\begin{problem}[2]
(Problem 2 stub)
\end{problem}
\begin{solution}
(Solution 2 stub)
\begin{enumerate}
\item (a)
\item (b)
\item (c)
\end{enumerate}
\end{solution}
\begin{problem}[3]
(Problem 3 stub)
\end{problem}
\begin{solution}
(Solution 3 stub)
\end{solution}
\end{document}


