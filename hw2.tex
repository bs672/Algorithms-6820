\documentclass{article}
\usepackage{amsmath}
\usepackage{amsthm}
\usepackage{algorithm, algpseudocode}
\usepackage{enumitem}
\newtheorem{lemma}{Lemma}

\author{Andrew Bennett, Richard Strong Bowen, Bhai Jaiveer Singh, Ivy Suiwen Wu}
\title{CS 6820 Problem Set 2}

\begin{document}
\maketitle

\section*{Problem 1}

In order to address this problem we will introduce a new directed graph $G^*$,
which contains all the vertices in $G$ as well as an additional source vertex $s$
and sink vertex $t$. For every edge $(u, v)$ in $G$ we introduce two directed
edges in $G^*$; one from $u$ to $v$ and one from $v$ to $u$. In addition we
introduce an edge from $s$ to every vertex from $V(G)$, and an edge from every
vertex from $V(G)$ to $t$. Finally we define capacities on these edges
as follows:

\[
c(u, v) =
    \begin{cases}
        m &\quad \text{if}\ u = s \\
        m + 2d - deg(u) &\quad \text{if}\ v = t \\
        1 &\quad \text{otherwise} \\
    \end{cases}
\]

Now, consider an arbitrary cut $(S, T)$ of the directed graph $G^*$. We will use the shorthand
$S' = S \setminus \{s\}$, and $T' = T \setminus \{t\}$, $V' = S' \cup T'$, and
$E' = \{(u, v) \mid (u, v) \in E(G^*),\ u \neq s,\ v \neq t\}$.
Furthermore, for every vertex $u \in V'$ we will define $k_{S'}(u)$, $k_{T'}(u)$, and $k_{V'}(u)$
to be the number of vertices in $S'$, $T'$, and $V'$ respectively that $u$ has a directed
edge to. Note that obviously for every vertex $k_{S'}(u) + k_{T'}(u) = k_{V'}(u)$.
The capacity of the cut is as follows:

\begin{align*}
  c(S, T) &= \sum_{u \in S'}{c(u, t)} + \sum_{v \in T'}{c(s, v)} + \sum_{u \in S',\ v \in T',\ (u, v) \in E'}{c(u, v)} \\
          &= \sum_{u \in S'}{(m + 2d - k_{V'}(u))} + \sum_{v \in T'}{m} + \sum_{u \in S',\ v \in T',\ (u, v) \in E'}{1} \\
          &= |S'|(m + 2d) - \sum_{u \in S'}{k_{V'}(u)} + |T'|m + \sum_{u \in S'}{k_{T'}(u)} \\
          &= |V'|m + 2|S'|d - \sum_{u \in S'}{k_{S'}(u)} \\
          &= |V'|m + 2|S'|d - 2|S'|D(S') \\ 
          &= |V'|m + 2|S'|(d - D(S')) \\ 
\end{align*}
where $m$ refers to the number of edges in the original graph $G$, and
$D(S')$ refers to the average degree of the vertices in the subgraph of $G$
defined by taking the vertices in $S'$, and all edges with both endpoints in $S'$.

Since $|V'|m$ is a constant for all cuts, the cut capacity will be minimized when
$|S'|(d - D(S'))$ is minimized. Furthermore, it is clear that the value of the cut
is less than $|V'|m$ if and only if the subgraph of $G$ defined by $S'$ has
average degree greater than $d$.

This gives a natural algorithm for finding a dense subgraph: run any max flow algorithm
on $G^*$ to calculate a maximum flow, and then compute a corresponding minimum
cut $c(S, T)$. If the value of the max flow / min cut is less than $|V'|m$ then $S'$ defines
a subgraph of $G$ with average degree greater than $d$. Otherwise no such dense subgraph
exists, because if there was one then it would give a smaller cut.

To analyze the running time of the algorithm, we need to specify the procedure used to
calculate the maximum flow and a corresponding minimum cut. For maximum flow calculation
we can use Dinitz's algorithm, which has a running time of $O(mn^2)$. For calculating
a minimum cut given this maximum flow, we calculate the residual graph $G^*_f$ using the
maximum flow $f$, and calculate the set $S'$ of all vertices reachable from $s$ in $G^*_f$.
As proven in lectures in the derivation of the max-flow min-cut theorem, this defines
a minimum cut. Calculating reachability can trivially be computed in $O(m + n)$. Therefore
the overall running time of our algorithm is polynomial.

\section*{Problem 2}
\subsection*{2a}
\subsection*{2b}
\subsection*{2c}
\subsection*{2d}

\section*{Problem 3}

\section*{Problem 4}

\end{document}
