\documentclass{article}
\usepackage{amsmath}
\usepackage{amssymb}
\usepackage{amsthm}
\usepackage{algorithm, algpseudocode}
\usepackage{enumitem}
\newtheorem{lemma}{Lemma}
\begin{document}
\begin{flushright}
Andrew Bennett\\
Richard Strong Bowen\\
Bhai Jaiveer Singh\\
Ivy Suiwen Wu\\
CS 6820 Problem Set 3
\end{flushright}
\section*{Problem 1}
\subsection*{1a}
\subsection*{1b}
Let $P = \{x \in \mathbb{R}^n | Ax \leqslant b\}$. We will construct $Q = \{y \in \mathbb{R}^{2n} | A'x \leqslant b'\} $ such that $Q$ is feasible if and only if $P$ is, by constructing $A'$ and $b'$. Let
\[ A' = \begin{bmatrix} A & -A \\ \multicolumn{2}{c}{-I_{2n}} \end{bmatrix}\]
(where $I_k$ means the $k \times k$ identity matrix) and
\[ b' = \begin{bmatrix} b \\ \vec0_{2n} \end{bmatrix} \] 
(where $\vec 0_k$ means the $k$-element all-zeros column vector). This transform is clearly polynomial-time, and ensures $Q$ is a subset of the positive orthant because of the last $2n$ constraints.

First we show that $P$ being feasible implies that $Q$ is. Suppose $x \in P$. We will produce a $y\in \mathbb{R}^{2n}$ which is in $Q$. Index $y$ as:
\[ y = \begin{bmatrix} y^+ \\ y^- \end{bmatrix} = \begin{bmatrix} y^+_1 \\ \vdots \\ y^+_n \\ y^-_1 \\ \vdots \\ y^-_n \end{bmatrix}\]
and set
\[y^+_i = \begin{cases} x_i & \quad \text {$x_i$ is nonnegative} \\ 0 & \quad \text {otherwise} \end{cases}\]
and
\[y^+_i = \begin{cases} -x_i & \quad \text {$x_i$ is negative} \\ 0 & \quad \text {otherwise} \end{cases}.\]
Note that $y \geqslant 0$ and $x = y^+ - y^-$. Then
\begin{align*}
A'y &= \begin{bmatrix} A & -A \\ \multicolumn{2}{c}{-I_{2n}} \end{bmatrix} \begin{bmatrix} y^+ \\ y^- \end{bmatrix} \\
&= \begin{bmatrix} Ax \\ -y \end{bmatrix} \\
&\leqslant \begin{bmatrix} b \\ \vec{0}_{2n} \end{bmatrix} \quad \text {(since $x \in P$ and $y$ is nonnegative)}\\
&= b'
\end{align*}
Hence $y \in Q$.

Second we show that $Q$ being feasible imples that $P$ is. Suppose $y\in Q$. We will produce an $x \in \mathbb{R}^{2n}$ which is in $P$. Index $y$ as above and let $x = y^+ - y^-$. Then we have this chain of implications
\begin{align*}
&\begin{bmatrix} A & -A \\ \multicolumn{2}{c}{-I_{2n}} \end{bmatrix} \begin{bmatrix} y^+ \\ y^- \end{bmatrix} \leqslant b' \quad \text{(since $y \in Q$)}\\
\implies & A(y^+ - y^-) \leqslant b \quad \text{(Ignoring the last $2n$ rows)}\\
\implies & Ax \leqslant b
\end{align*}
Hence $x \in Q$.

\subsection*{1c}
\subsection*{1d}
\subsection*{1e}
\subsection*{1f}
\section*{Problem 2}
\subsection*{2a}
\subsection*{2b}
\end{document}
