\documentclass{article}
\usepackage{amsmath}
\usepackage{amssymb}
\usepackage{amsthm}
\usepackage{algorithm, algpseudocode}
\usepackage{enumitem}
\newtheorem{lemma}{Lemma}
\begin{document}
\begin{flushright}
Andrew Bennett\\
Richard Strong Bowen\\
Bhai Jaiveer Singh\\
Ivy Suiwen Wu\\
CS 6820 Problem Set 3
\end{flushright}
\section*{Problem 1}
\subsection*{1a}
\begin{lemma} For an integer-value invertible $n\times n$ matrix $T$, $|\text{det}(T)| \geq 1$ and  $|\text{det}(T)| \leq (nU)^n$ \end{lemma}\begin{proof} For a $1\times 1$ matrix $T = [x]$, we have $|\text{det}(T)|= |x|\leq |x|$ and $|\text{det}(T)| = |x| \geq 1$ because $T$ is integer-valued. For the induction step, suppose the statement is true for matrices whose sizes are $n\times n$,$n\in\{1,2,3..k\}$. Consider an invertible $(k+1)\times(k+1)$ matrix T. 
\begin{align*}
\text{det}(T) &= \sum_{i=1}^{k+1} (-1)^{i-1}\text{det}(A_i)\cdot T_{1i}\\
|\text{det}(T)|&\leq |\sum_{i=1}^{k+1} (-1)^{i-1}\text{det}(A_i)\cdot T_{1i}|\\
&\leq \sum_{i=1}^{k+1}(kU)^{n}\cdot |T_{1i}|\\
&\leq(k+1)U\cdot (kU)^{k}\\
&\leq ((k+1)U)^{k+1}
\end{align*}
where $A_i$ is an $n\times n$ matrix formed by removing row $1$ and column $i$ of $T$. $|\text{det}(T)|\geq 1$ follows from the fact that $T$ is integer-valued and so are its sub-matrices. \end{proof}
We have $T^{-1} = \frac{1}{\text{det}(T)}C^{\intercal}$, where $C$ is the cofactor matrix of $T$. Therefore, we know the absolute value of elements of $C$ is bounded between $[1, (nU)^{n}]$. From the lemma above, we also have $\frac{1}{|\text{det}(T)|}$ is bounded between $[(nU)^{-n},1]$. This concludes the proof that the absolute value of entries of $T^{-1}$ is bounded above by $(nU)^{n}$ and the absolute value of non-zero entries of $T^{-1}$ is bounded below by $(nU)^{-n}$.
\subsection*{1b}
Let $P = \{x \in \mathbb{R}^n | Ax \leqslant b\}$. We will construct $Q = \{y \in \mathbb{R}^{2n} | A'x \leqslant b'\} $ such that $Q$ is feasible if and only if $P$ is, by constructing $A'$ and $b'$. Let
\[ A' = \begin{bmatrix} A & -A \\ \multicolumn{2}{c}{-I_{2n}} \end{bmatrix}\]
(where $I_k$ means the $k \times k$ identity matrix) and
\[ b' = \begin{bmatrix} b \\ \vec0_{2n} \end{bmatrix} \] 
(where $\vec 0_k$ means the $k$-element all-zeros column vector). This transform is clearly polynomial-time, and ensures $Q$ is a subset of the positive orthant because of the last $2n$ constraints.

First we show that $P$ being feasible implies that $Q$ is. Suppose $x \in P$. We will produce a $y\in \mathbb{R}^{2n}$ which is in $Q$. Index $y$ as:
\[ y = \begin{bmatrix} y^+ \\ y^- \end{bmatrix} = \begin{bmatrix} y^+_1 \\ \vdots \\ y^+_n \\ y^-_1 \\ \vdots \\ y^-_n \end{bmatrix}\]
and set
\[y^+_i = \begin{cases} x_i & \quad \text {$x_i$ is nonnegative} \\ 0 & \quad \text {otherwise} \end{cases}\]
and
\[y^+_i = \begin{cases} -x_i & \quad \text {$x_i$ is negative} \\ 0 & \quad \text {otherwise} \end{cases}.\]
Note that $y \geqslant 0$ and $x = y^+ - y^-$. Then
\begin{align*}
A'y &= \begin{bmatrix} A & -A \\ \multicolumn{2}{c}{-I_{2n}} \end{bmatrix} \begin{bmatrix} y^+ \\ y^- \end{bmatrix} \\
&= \begin{bmatrix} Ax \\ -y \end{bmatrix} \\
&\leqslant \begin{bmatrix} b \\ \vec{0}_{2n} \end{bmatrix} \quad \text {(since $x \in P$ and $y$ is nonnegative)}\\
&= b'
\end{align*}
Hence $y \in Q$.

Second we show that $Q$ being feasible imples that $P$ is. Suppose $y\in Q$. We will produce an $x \in \mathbb{R}^{2n}$ which is in $P$. Index $y$ as above and let $x = y^+ - y^-$. Then we have this chain of implications
\begin{align*}
&\begin{bmatrix} A & -A \\ \multicolumn{2}{c}{-I_{2n}} \end{bmatrix} \begin{bmatrix} y^+ \\ y^- \end{bmatrix} \leqslant b' \quad \text{(since $y \in Q$)}\\
\implies & A(y^+ - y^-) \leqslant b \quad \text{(Ignoring the last $2n$ rows)}\\
\implies & Ax \leqslant b
\end{align*}
Hence $x \in Q$.

\subsection*{1c}

\begin{lemma}
  Suppose the polyhedron $P = \{x \in \mathbb{R}^n | Ax \leq b\}$ is non-empty,
  and there are at least as many constraints as variables. Then
  there is a solution $x \in P$ that is tight in at least $n$
  constraints, where $n$ is the number of variables.
\end{lemma}
\begin{proof}
  First we can observe that if $P$ is non-empty there is obviously
  a solution that makes at least one constraint tight (assuming there
  is at least one constraint), because from any feasible solution we
  can move towards the decision boundary defined by any arbitrary
  constraint until we reach it (or alternatively reach the decision
  boundary of some other constraint first.) This gives a
  solution that is tight in one constraint.

  Given this factoid we can show the general result via induction
  on $n$: the case where $n = 1$ follows trivially from the prior
  factoid. Otherwise, as argued above, we can obtain a feasible
  solution where at least one constraint is tight. We can use
  the equation corresponding to this tight constraint to eliminate
  one of the variables, giving a smaller feasibility problem with
  one fewer variable and one fewer constraint. Note that by
  construction this smaller problem is also non-empty, since at minimum
  it contains the above feasible solution. By the inductive hypothesis
  we can find a feasible solution to this sub-problem where $n - 1$
  constraints are tight, which trivially gives a feasible solution
  to the original feasibility problem with $n$ constraints tight
  (by re-introducing the eliminated constraint, and setting the
  eliminated variable according to that constraint). QED.
\end{proof}

% We can first observe that we only need to prove the forward direction
% ($P$ is non-empty implies that $P \cap \[0, R\]^n$ is non-empty),
% since the reverse is trivial. Suppose that $P$ is non-empty. First
% we can observe we can trivially ensure we have at least $n$ constraints
% without changing $P$, by making sure that the non-negativity constraints
% for all variables are explicitly included.

% Given the above lemma,
% there is a feasible solution $x^*$ that is tight in at least $n$
% of the constraints. Let $A^*$ be the $n \times n$ matrix from the rows
% of $A$ corresponding to $n$ such tight constraints (chosen
% arbitrarily if there are more than $n$ tight constraints), and let
% $b^*$ be the column vector from the corresponding rows of $b$.
% Then by construction the following equations are satisifed:


\subsection*{1d}
\subsection*{1e}
\subsection*{1f}
\section*{Problem 2}
\subsection*{2a}
\subsection*{2b}
\end{document}
