\section*{Problem 3}

{\bf Proof that FRATERNITY is in NP}

Given a problem input of FRATERNITY with a list of members, parameter
k, and f(x), s(x), j(x) for every member x, a solution set S can be
verified in polynomial time with the following efficient verifier:

\begin{algorithm}[H]
\caption{}
\begin{algorithmic}[1]
\For {each x in S}
\For {each y in S}
\If {f(x) = y or s(x) = y or j(x) = y}
\State Return NO
\EndIf
\EndFor
\EndFor
\State Return YES
\end{algorithmic}
\end{algorithm}
The inner loop has k iterations, the outer loop has k
iterations. So the running time of the verifier is O($k^2$)\\
\\
{\bf Proof that FRATERNITY is NP-Complete}\\
\\
We will prove that the FRATERNITY problem is NP-Complete by a
reduction from IND SET, i.e. prove IND SET $\leq_p$ FRATERNITY.

Consider an IND SET problem with inputs \{G=(V,E), k\}. From this input, construct an input of FRATERNITY as follows:
For each node v $\in$ V, create a node gadget called {\bf gadget(v)}. If D is the maximum degree of G, each gadget(v) consists of 2D-1 fraternity members labelled, $v_1$...$v_{2D-1}$. For all $v_i, v_{i+1}$ in gadget(v):

\[
v_i =
    \begin{cases}
        f(v_i+1) &\quad \text{if $i$ is odd} \\
        s(v_i+1) &\quad \text{otherwise} \\
    \end{cases}
\]

For every edge $(u,v) \in$ E connecting nodes u and v, pick some members $u_a$, $v_b$ from gadget(u) and gadget(v) respectively, such that $j(u_a) = j(v_b) = NULL$ \textbf{and} a and b are both odd. Now assign $u_a$ = $j(v_b)$.\\

\textbf{Claim:} There will always exist such $u_a$, $v_b$ for each edge $(u,v) \in E$

\textbf{Proof:} This must be true because if the maximum degree is D, then for any node v $\in$ V, the maximum number of edges $(u,v) \in$ E $\leq$ D. Hence, we will need at most D such $v_b$'s. There are exactly D number of odd numbered members in each gadget, hence there will always be a $v_b$ such that b is odd and $j(v_b) = NULL$ for any edge $(e,v) \in$ E. This is symmetrically true for node u.\\

\textbf{Claim:} The FRATERNITY input construction can be done in polynomial time.

\textbf{Proof:} The maximum degree D $< |E|$.
Creating each node gadget is done in O(D) time. There are $|V|$ such node gadgets. Assigning the f(x) and s(x) relationships is also O(D). Assigning the j(x) relationship can be done in O($|E|$) time. So overall, the construction can be done in time polynomial to the size of inputs V and E.\\
\\
We will prove that for any input to IND SET and a correspondingly constructed input to FRATERNITY, an IND SET of size k exists iff a committee of size kD+(D-1)($|V|$-k) exists
\\

\textbf{Lemma 1:} If an IND SET of size k exists, then FRATERNITY problem committee of size kD + (D-1)($|V|$-k) exists for the correspondingly constructed input

\textbf{Proof:}
Given the IND SET input G=(V,E) and the Independent set S of size k, we choose the committee members as follows:

If node v $\in$ S, then choose all the odd numbered members of gadget(v)

If node v $\notin$ S, then choose all the even numbered members of gadget(v)

Each gadget(v) for v $\in$ S provides D members, each gadget(v) for v $\notin$ S provides D-1 members for a total of kD + (D-1)($|V|$-k) members.

This is a valid solution to the fraternity problem. For any given gadget, it is never the case that both an odd numbered member and an even numbered member are chosen, so we never choose two members x, y such that f(x) = y or s(x) = y. The j(x) constraint is also enforced. We can prove this by contradiction. Consider the case that two members $x_a$ and $y_b$ are chosen such that j($x_a$) = $y_b$. For this assignment to exist, two things must be true - there must be an edge (x,y) in E, and $x_a$ and $y_b$ must both be odd numbered members of gadget(x) and gadget(y) respectively. Because edge(x,y) is in E, both nodes x and y cannot be in S. Hence for at least one of them we only picked even numbered members to be in the committee. This is a contradiction, since both $x_a$ and $y_b$ were supposed to be odd numbered members. So this is a valid solution to the fraternity problem.\\
\\
\textbf{Lemma 2:} If FRATERNITY problem committee of size kD + (D-1)($|V|$-k) exists, then an IND SET of size k exists

\textbf{Proof:}
The Independent set S of size at least k is picked as follows:

For every node gadget gadget(v) that contributes D members, put its corresponding node v in the independent set.

There must be at least k gadgets that contribute D members. Suppose it were the case that some c $<$ k gadgets contributed D members. Then the size of the committee would be cD + ($|V|$-c)(D-1) $<$ kD + (D-1)($|V|$-k). Since at least k gadgets contribute D members, the independent set has size at least k.

No two nodes of the independent set are connected. We will prove this by contradiction. Suppose that two connected nodes u, v exist in S. Then it must be that gadget(u) and gadget(v) both contributed D members to the committee. For them both to contribute D members, all of their odd numbered members (and none of their even numbered members) must be part of the committee. However, if all of their odd numbered members were chosen, and there exists the edge (u,v), then it must be the case that we chose two members $u_a$ , $u_b$ such that j($u_a)$ = $u_b$. This contradicts the assumption that we started with a valid committee. Hence this is a valid independent set solution of size at least k.\\

Lemmas 1 and 2 together complete our proof that IND SET $\leq_p$ FRATERNITY
